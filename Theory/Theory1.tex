\documentclass[12pt]{article}
\usepackage[utf8]{inputenc}
\usepackage{amsmath,amssymb,amsthm}
\newenvironment{note}{\paragraph{NOTE:}}{}
\newenvironment{field}{\paragraph{field:}}{}
\newenvironment{tags}{\paragraph{tags:}}{}

\begin{document}



%%start_tag Casella Berger
\tags{Theory1}
%%start_tag Casella Ch1
\begin{note}
  \begin{field}
    \begin{tabular}{|c| c c |}
    \hline \\
    & replace & no replacement \\
    number of trials &  &  \\
    Draw till nth success & & \\
    \hline
    \end{tabular}
  \end{field}
  \begin{field}
    \begin{tabular}{|c| c c |}
    \hline \\
    & replace & no replacement \\
    number of trials & Binom & Hypergeometric \\
    Draw till nth success & Nbinom & Negative hypergeometric \\
    \hline
    \end{tabular}
  \end{field}
\end{note}

\begin{note}
  \begin{field}
    Plug uniform into inverse CDF
  \end{field}
  \begin{field}
    Get cdf
  \end{field}
\end{note}

\begin{note}
  \begin{field}
    Sample Space
  \end{field}
  \begin{field}
    The set, $S$, of all possible outcomes of a particular experiment is called the \textit{sample space} for the experiment.
  \end{field}
\end{note}

\begin{note}
  \begin{field}
    Event
  \end{field}
  \begin{field}
    An \textit{event} is any collection of possible outcomes of an experiment, that is, any subset of $S$ (including $S$ itself).
  \end{field}
\end{note}

\begin{note}
  \begin{field}
    Union
  \end{field}
  \begin{field}
    $ A \cup B = \{x:x \in A \text{ or } x \in B\}$
  \end{field}
\end{note}

\begin{note}
  \begin{field}
    Intersection
  \end{field}
  \begin{field}
    $A \cap B = \{x:x \in A \text{ and } x \in B\}$
  \end{field}
\end{note}

\begin{note}
  \begin{field}
    Complementation
  \end{field}
  \begin{field}
    $A^c = \{x:x\notin A\}$
  \end{field}
\end{note}

\begin{note}
  \begin{field}
    Commutativity
    $$ A \cup B = $$
    $$ A \cap B = $$
  \end{field}
  \begin{field}
    Commutativity
    $$ A \cup B = B \cup A$$
    $$ A \cap B = B \cap A$$
  \end{field}
\end{note}

\begin{note}
  \begin{field}
    Associativity
    $$ A \cup (B \cup C) = $$
    $$ A \cap (B \cap C) = $$
  \end{field}
    \begin{field}
      Associativity
        $$ A \cup (B \cup C) = (A \cup B) \cup C$$
        $$ A \cap (B \cap C) = (A \cap B) \cap C$$
    \end{field}
\end{note}

\begin{note}
  \begin{field}
    Distributive Laws
    $$A \cap ( B \cup C) = $$
    $$ A \cup ( B \cap C) = $$
  \end{field}
    \begin{field}
      Distributive Laws
        $$A \cap ( B \cup C) = (A \cap B) \cup (A \cap C)$$
        $$ A \cup ( B \cap C) = (A \cup B) \cap ( A \cup C)$$
    \end{field}
\end{note}

\begin{note}
  \begin{field}
    DeMorgan's Laws
    $$(A \cup B)^c = $$
    $$(A \cap B)^c = $$
  \end{field}
    \begin{field}
      DeMorgan's Laws
        $$(A \cup B)^c = A^c \cap B^c$$
        $$(A \cap B)^c = A^c \cup B^c$$
    \end{field}
\end{note}

\begin{note}
  \begin{field}
    Disjoint
  \end{field}
    \begin{field}
        Disjoint: Two events $A$ and $B$ are disjoint ( or mutually exclusive) if $ A \cap B = \emptyset $
    \end{field}
\end{note}

\begin{note}
  \begin{field}
    $$P(A_1 \cap A_2 \cap \cdots \cap A_n)  = $$
  \end{field}
  \begin{field}
    $$P(A_1)P(A_2|A_1)P(A_3|A_1A_2) \ldots P(A_n|A_1\cdots A_{n-1}) $$
  \end{field}
\end{note}

\begin{note}
  \begin{field}
    $$ P(A,B,C) = $$
  \end{field}
  \begin{field}
    $$ P(A,B,C) = P(A)P(B|A)P(C|A,B)$$
  \end{field}
\end{note}

\begin{note}
  \begin{field}
    $$P(A \cup B \cup C) =  $$
  \end{field}
  \begin{field}
    $$P(A \cup B \cup C) =  P(A) + P(B) + P(C) - P(A \cap B) - P(B\cap C) - P(A \cap C) + P(A \cap B \cap C)$$
  \end{field}
\end{note}



\begin{note}
  \begin{field}
    Pairwise disjoint
  \end{field}
    \begin{field}
        Two Events $A_1, A_2$ are pairwise disjoint ( or mutually exclusive) if $A_i \cap A_j = \emptyset $ for all $i \neq j$
    \end{field}
\end{note}

\begin{note}
    \begin{field}
      Partition
    \end{field}
    \begin{field}
        If $A_1, A_2, \ldots$ are pairwise disjoint and $\cup_{i=1}^\infty A_i = S$, then the collection $A_1, A_2, \ldots$ forms a partition of $S$.
    \end{field}
\end{note}

\begin{note}
    \begin{field}
        Sigma Algebra
    \end{field}
    \begin{field}
        A collection of subsets of $S$ is called a sigma algebra (or Borel field), denoted by $\mathcal{B}$, if it satisfies the following three properties:
            \begin{enumerate}
              \item $\emptyset \in \mathcal{B}$ (the empty set is an element of $\mathcal{B}$)
              \item If $A \in \mathcal{B}$, then $A^c \in \mathcal{B}$ ($\mathcal{B}$ is closed under complementation)
              \item If $A_1, A_2, \ldots \in \mathcal{B}$, then $\cup_{i-1}^\infty A_i \in \mathcal{B} \mathcal{B}$ is closed under countable unions)
            \end{enumerate}
    \end{field}
\end{note}

\begin{note}
    \begin{field}
        Probability Function / Kolmogorov Axioms
    \end{field}
    \begin{field}
        Given a sample space $S$ and an associated sigma algebra $\mathcal{B}$, a probability function is a function $P$ with domain $\mathcal{B}$ that satisfies:
            \begin{enumerate}
              \item $P(A) \geq 0$ for all $A \in \mathcal{B}$
              \item $P(S) = 1$
              \item If $A_1, A_2, \ldots \mathcal{B}$ are pairwise disjoint, then $P(\cup_{i=1}^\infty A_i) = \sum_{i=1}^\infty P(A_i)$ (Axiom of Countable Additivity)
            \end{enumerate}
    \end{field}
\end{note}

\begin{note}
    \begin{field}
      If $A\in \mathcal{B}$ and $B \in \mathcal{B}$ are disjoint, then
          $$P(A\cup B) = $$
        Axiom of Finite Additivity
    \end{field}
    \begin{field}
        If $A\in \mathcal{B}$ and $B \in \mathcal{B}$ are disjoint, then
            $$P(A\cup B) = P(A) + P(B) $$
    \end{field}
\end{note}

\begin{note}
    \begin{field}
        Properties of probability functions
        \begin{enumerate}
              \item $P(\emptyset) = $
              \item $P(A) $
              \item $P(A^c) = $
            \end{enumerate}
    \end{field}
    \begin{field}
      Properties of probability functions
        \begin{enumerate}
              \item $P(\emptyset) = 0$
              \item $P(A) \leq 1$
              \item $P(A^c) = 1 - P(A)$
            \end{enumerate}
    \end{field}
\end{note}

\begin{note}
    \begin{field}
        If $P$ is a probability function and $A$ and $B$ are any sets in $\mathcal{B}$, then $$P(B\cap A^c) = $$
    \end{field}
    \begin{field}
        If $P$ is a probability function and $A$ and $B$ are any sets in $\mathcal{B}$, then $$P(B\cap A^c) = P(B) - P(A \cap B)$$
    \end{field}
\end{note}

\begin{note}
    \begin{field}
        If $P$ is a probability function and $A$ and $B$ are any sets in $\mathcal{B}$, then $$P(A\cup B) = $$
    \end{field}
    \begin{field}
        If $P$ is a probability function and $A$ and $B$ are any sets in $\mathcal{B}$, then $$P(A\cup B) = P(A) + P(B) - P(A \cap B)$$
    \end{field}
\end{note}

\begin{note}
    \begin{field}
        If $P$ is a probability function and $A$ and $B$ are any sets in $\mathcal{B}$, then if $A \subset B$ then
    \end{field}
    \begin{field}
        If $P$ is a probability function and $A$ and $B$ are any sets in $\mathcal{B}$, then if $A \subset B$ then $P(A) \leq P(B)$
    \end{field}
\end{note}

\begin{note}
    \begin{field}
        Bonferroni's Inequality
        $$P(A\cap B)  $$
    \end{field}
    \begin{field}
      Bonferroni's Inequality:
        $$P(A\cap B) \geq P(A) + P(B) -1 $$
    \end{field}
\end{note}

\begin{note}
    \begin{field}
        If $P$ is a probability function, then for any partition $C_1, C_2, \ldots P(A) = $
    \end{field}
    \begin{field}
        If $P$ is a probability function, then for any partition $C_1, C_2, \ldots P(A) = \sum_{i=1}^\infty P(A \cap C_i)$
    \end{field}
\end{note}

\begin{note}
    \begin{field}
        Boole's Inequality
        $$P(\cup_{i=1}^\infty A_i)$$
    \end{field}
    \begin{field}
        If $P$ is a probability function,
            $$P(\cup_{i=1}^\infty A_i) \leq \sum_{i=1}^\infty P(A_i) \text{ for any sets } A_1, A_2, \ldots $$
    \end{field}
\end{note}

\begin{note}
    \begin{field}
        Fundamental Theorem of Counting
    \end{field}
    \begin{field}
        Ifa job consists of $k$ separate tasks, the $i$th of which can be done in $n_i$ ways, $i = 1, \ldots, k$, then the entire job can be done in $n_1 \times n_2 \times \cdots, \times n_k$ ways.
    \end{field}
\end{note}

\begin{note}
    \begin{field}
        Ordered without replacement: number of arrangements of size $r$ from $n$ objects
    \end{field}
    \begin{field}
        $$\frac{n!}{(n-r)!}$$

        eg lottery with $n=44$ choices for $r=6$ values, cant use same number twice, order matters
    \end{field}
\end{note}

\begin{note}
    \begin{field}
        Unordered without replacement: number of arrangements of size $r$ from $n$ objects
      \end{field}
    \begin{field}
        $$\binom{n}{r} = \frac{n!}{r!(n-r!)}$$

        eg lottery with $n=44$ choices for $r=6$ values, cant use same number twice, order does not matter (Use ordered without replacement and divide by redundant orderings )
    \end{field}
\end{note}


\begin{note}
    \begin{field}
        Ordered with replacement: number of arrangements of size $r$ from $n$ objects
    \end{field}
    \begin{field}
      Ordered with replacement: number of arrangements of size $r$ from $n$ objects
        $$n^r$$
        eg lottery with $n=44$ choices for $r=6$ values, can use same number twice, order matters
    \end{field}
\end{note}

\begin{note}
    \begin{field}
        Unordered with replacement: number of arrangements of size $r$ from $n$ objects
    \end{field}
    \begin{field}
      Unordered with replacement: number of arrangements of size $r$ from $n$ objects
        $$\binom{n+r-1}{r} = \frac{(n+r-1)!}{r!(n-1)!}$$

        eg lottery with $n=44$ choices for $r=6$ values, can use same number twice, order does not matters
    \end{field}
\end{note}

\begin{note}
  \begin{field}
    Number of arrangements of size $r$ from $n$ objects

    \begin{tabular}{|c| c c|}
      \hline \\
       & Without Replacement & With replacement\\
       \hline \\
       Ordered & & \\
       Unordered & & \\
       \hline
    \end{tabular}
  \end{field}
  \begin{field}
    Number of arrangements of size $r$ from $n$ objects

    \begin{tabular}{|c| c c|}
      \hline \\
       & Without Replacement & With replacement\\
       \hline \\
       Ordered & $\frac{n!}{(n-r)!}$ & $n^r$ \\
       Unordered & $\binom{n}{r}$& $\binom{n+r-1}{r}$ \\
       \hline
    \end{tabular}
  \end{field}
\end{note}



\begin{note}
    \begin{field}
        Binomial Coefficient $\binom{n}{r}$
    \end{field}
    \begin{field}
      Binomial Coefficient
        $$\binom{n}{r} = \frac{n!}{r!(n-r)!}$$
    \end{field}
\end{note}

\begin{note}
  \begin{field}
    $$P(A|B)=$$
  \end{field}
  \begin{field}
    $$ P(A|B) = \frac{P(A \cap B)}{P(B)} $$
  \end{field}
\end{note}

\begin{note}
  \begin{field}
    Statistically independent
    $P(A \cap B) = $
  \end{field}
  \begin{field}
    Statistically independent
    $P(A \cap B) = P(A)P(B)$
  \end{field}
\end{note}

\begin{note}
  \begin{field}
    If $A$ and $B$ are independent events, what else is independent?
  \end{field}
  \begin{field}
    \begin{itemize}
      \item $A $ and $B^c$
      \item $A^c$ and $B$
      \item $A^c$ and $B^c$
    \end{itemize}
  \end{field}
\end{note}

\begin{note}
  \begin{field}
    Mutually independent
  \end{field}
  \begin{field}
    A collection of events $A_1, \ldots , A_n$ are mutually independent for any subcollection $A_{i1}, \ldots , A_{ik}$, we have
    $$ P((\cap_{j=1}^k A_{ij})) = \prod_{j=1}^k P(A_{ij}) $$
  \end{field}
\end{note}

\begin{note}
  \begin{field}
    Random variable
  \end{field}
  \begin{field}
    A random variable is a function from a sample space $S$ into the real numbers
  \end{field}
\end{note}

\begin{note}
  \begin{field}
    Definition of a pdf
  \end{field}
  \begin{field}
    A function $f_X(x)$ is a pdf (or pmf) of a random variable $X$ if and only if
    \begin{enumerate}
      \item $f_x(x) \geq 0$ for all $x$
      \item $\sum_x f_x(x) = 1$ or $\int_{-\infty}^\infty f_x(x) dx = 1$
    \end{enumerate}
  \end{field}
\end{note}


%%end_tag
%%start_tag Casella Ch2
\begin{note}
  \begin{field}
    (Theorem) Let $X$ have cdf $F_X(x)$, let $Y = g(X)$
    \begin{enumerate}
      \item If $g$ is an increasing function on $X$, $F_Y(y) = $ for $y \in Y$
      \item If $g$ is a decreasing function on $X$ and $X$ is a continuous random variable, $F_y(y) = $ for $y \in Y$
    \end{enumerate}
  \end{field}
  \begin{field}
    (Theorem) Let $X$ have cdf $F_X(x)$, let $Y = g(X)$
    \begin{enumerate}
      \item If $g$ is an increasing function on $X$, $F_Y(y) = F_X(g^{-1}(y))$ for $y \in Y$
      \item If $g$ is a decreasing function on $X$ and $X$ is a continuous random variable, $F_y(y) = 1 - F_X(g^{-1}(y))$ for $y \in Y$
    \end{enumerate}
  \end{field}
\end{note}


\begin{note}
  \begin{field}
    Method of pdf
  \end{field}
  \begin{field}
    Conditions:
    \begin{enumerate}
      \item $g$ is a monotone function
      \item $f_X(x)$ is continuous on $X$
      \item $g^{-1}(y)$ has a continuous derivative
    \end{enumerate}
    Let $X$ have pdf $f_x(x)$ and let $Y = g(Y)$

    $$f_Y(y) = f_x(g^{-1}(y))\big|\frac{d}{dy}g^{-1}(y)$$
  \end{field}
\end{note}

\begin{note}
  \begin{field}
    (Theorem) Let $X$ have cdf $F_X(x)$, let $Y = g(X)$
    \begin{itemize}
      \item If $g$ is an increasing function, $F_Y(y) = $
      \item If $g$ is a decreasing function, and $X$ is a continuous random variable, $F_Y(y) = $
    \end{itemize}
  \end{field}
  \begin{field}
    (Theorem) Let $X$ have cdf $F_X(x)$, let $Y = g(X)$
    \begin{itemize}
      \item If $g$ is an increasing function, $F_Y(y) = F_x(g^{-1}(y))$
      \item If $g$ is a decreasing function, and $X$ is a continuous random variable, $F_Y(y) = 1 - F_X(g^{-1}(y))$
    \end{itemize}
  \end{field}
\end{note}

\begin{note}
  \begin{field}
    Method of CDF eg: $X \sim Unif(0,1)$, $Y = -log(X)$
    $F_Y(y) = $
  \end{field}
  \begin{field}
    $F_Y(y) = 1 - F_x(g^{-1}(y)) = 1 - F_X(e^{-y}) = 1 - e^{-y}$
  \end{field}
\end{note}


\begin{note}
  \begin{field}
    $X$ is a continuous random variable. For $y > 0$,
    $F_Y(y) = $
  \end{field}
  \begin{field}
    \begin{align*}
      F_Y(y) &= P(Y \leq y)\\
      &= P(X^2 \leq y)\\
      &= P(- \sqrt{y} \leq X \leq \sqrt{y})\\
      &= P(X \leq \sqrt{y}) - P(X \leq -\sqrt{y})\\
      &= F_X(\sqrt{y}) - F_X( - \sqrt{y})
    \end{align*}
  \end{field}
\end{note}

\begin{note}
  \begin{field}
    Pdf of $F_X(g(X))$, where $Y = g(X)$
  \end{field}
  \begin{field}
    Chain rule:
    $f_Y(y) = g'(y)f(g(y))$
  \end{field}
\end{note}

\begin{note}
  \begin{field}
    Method of pdf if $g$ is not monotone all entire domain
  \end{field}
  \begin{field}
    $f_Y = \sum f_x(g_i^{-1}(y))|\frac{d}{dy}g_i^{-1}(y)|$ $y \in Y$, 0 otherwise

    eg: $Y = X^2$,
  \end{field}
\end{note}

\begin{note}
  \begin{field}
    $P(Y  \leq y)$ when $Y = F_X(x)$
  \end{field}
  \begin{field}
    \begin{align*}
      P(Y \leq y) &= P(X \leq F_x^{-1}(y))\\
      &= F_X(F_X^{-1}(y))\\
      &= y
    \end{align*}
    $Y$ is uniformly distributed
  \end{field}
\end{note}

\begin{note}
  \begin{field}
    $M_x(t) = $ (discrete )
  \end{field}
  \begin{field}
    $M_x(t) = E(e^{tX}) = \sum_x e^{tX}P(X)$ (discrete )
  \end{field}
\end{note}

\begin{note}
  \begin{field}
    $M_x(t) = $ (continuous)
  \end{field}
  \begin{field}
    $M_x(t) = E(e^{tX}) = \int_{-\infty}^\infty e^{tX}f_x(x)dx$ (continuous)
  \end{field}
\end{note}

\begin{note}
  \begin{field}
    $E(X^n) = $
  \end{field}
  \begin{field}
    $E(X^n) = M_x^n(0) = \frac{d^n}{dt^n}M_x(t)|_{t=0}$
  \end{field}
\end{note}

\begin{note}
  \begin{field}
    $M(aX + b)(t) = $
  \end{field}
  \begin{field}
    $M(aX + b)(t) = e^{bt}M_x(at)$
  \end{field}
\end{note}

\begin{note}
  \begin{field}
    If $E(X^n)$ exists then...
  \end{field}
  \begin{field}
    If $E(X^n)$ exists then $E(X^m)$ exists for $m \leq n$
  \end{field}
\end{note}

\begin{note}
  \begin{field}
    If $X_i$ are independent and $Y = a_1X_1 + \cdots + a_nX_n + b$, then $M_Y(t) = $
  \end{field}
  \begin{field}
    If $X_i$ are independent and $Y = a_1X_1 + \cdots + a_nX_n$, then $M_Y(t) = e^{bt}\prod_{i=1}^n M_{X_i}(a_i t) $
  \end{field}
\end{note}

\begin{note}
  \begin{field}
    Example of using MGF for finding expected value:
    MGF gamma: $(\frac{1}{1 - \beta t})^\alpha$: $E(X) = $
  \end{field}
  \begin{field}
    $E(X) = \frac{\alpha\beta}{(1 - \beta t)^{\alpha+1}}|_{t=0} = \alpha \beta$
  \end{field}
\end{note}

\begin{note}
  \begin{field}
    Using MGF to relate distributions: MGF exp = $( 1 - \beta t)^{-1}$
  \end{field}
  \begin{field}
    $Y = \sum X_i$ is gamma as MGF gamma is $(1 - \beta t)^{-\alpha}$
  \end{field}
\end{note}

\begin{note}
  \begin{field}
    First step in transforming a RV
  \end{field}
  \begin{field}
    Determine support
  \end{field}
\end{note}

\begin{note}
  \begin{field}
    $nth$ Moment of X
  \end{field}
  \begin{field}
    $E(X^n)$
  \end{field}
\end{note}

\begin{note}
  \begin{field}
    $n$th central moment of $X$
  \end{field}
  \begin{field}
    $E(X - \mu)^n$
  \end{field}
\end{note}

\begin{note}
  \begin{field}
    $(a + b)^n = $
  \end{field}
  \begin{field}
    $(a + b)^n = \sum_{x = 0}^n \binom{n}{x}a^xb^{n-x}$
  \end{field}
\end{note}

\begin{note}
  \begin{field}
    $\sum_{x = 0}^n \binom{n}{x}a^xb^{n-x} = $
  \end{field}
  \begin{field}
    $(a + b)^n$
  \end{field}
\end{note}

%%end_tag
%%start_tag Casella Ch3
\begin{note}
  \begin{field}
    $N$ balls $r$ red $N - r$ green. Select $n$ balls. Probability that $y$ are red?
  \end{field}
  \begin{field}
    Hypergeometric distribution($N,r,n$)
  \end{field}
\end{note}

\begin{note}
  \begin{field}
    Hypergeometric distribution description $(N,r,n)$
  \end{field}
  \begin{field}
    $N$ is total balls, $r$ is number red balls, $n$ is number balls selected.
  \end{field}
\end{note}

\begin{note}
  \begin{field}
    Negative binomial description
  \end{field}
  \begin{field}
    Number of Bernoulli trials required to get a fixed number of successes. $r$ being the $r$th success
  \end{field}
\end{note}

\begin{note}
  \begin{field}
    Geometric description
  \end{field}
  \begin{field}
    Modeling waiting time. $X$ is the trial at which the first success occurs.
  \end{field}
\end{note}

\begin{note}
  \begin{field}
    Location-scale family for $f(x)$
  \end{field}
  \begin{field}
    $1/\sigma f((x - \mu)/\sigma)$
  \end{field}
\end{note}

\begin{note}
  \begin{field}
    Given $X$, give the mean and variance for the location-scale random variable
$$Y= \frac{1}{\sigma} f\bigg(\frac{(y - \mu)}{\sigma}\bigg)$$
$E(Y) = $\\
$V(Y) = $
  \end{field}
  \begin{field}
    $E(Y) = \sigma E(X) + \mu$, $V(Y) = \sigma^2 V(X)$
  \end{field}
\end{note}


\begin{note}
  \begin{field}
    $X \sim Pois(\lambda)$ $P(X = x+1) = $
  \end{field}
  \begin{field}
    $X \sim Pois(\lambda)$ $P(X = x+1) = \frac{\lambda}{x + 1}P(X = x)$
  \end{field}
\end{note}

%%end_tag
%%start_tag Casella Ch4

\begin{note}
  \begin{field}
    $f(y|x) = $
  \end{field}
  \begin{field}
    $f(y|x) = \frac{f(x,y)}{f_x(x)}$
  \end{field}
\end{note}

\begin{note}
  \begin{field}
    $E(g(Y)|x) = $
  \end{field}
  \begin{field}
    $E(g(Y)|x) = \int_{-\infty}^\infty g(y)f(y|x) dy $
  \end{field}
\end{note}

\begin{note}
  \begin{field}
    Example of calculating donditional pdfs
$f(x,y) = e^{-y}, 0 < x < y < \infty$. $f(y|x) = $

  \end{field}
  \begin{field}
    \begin{align*}
      f_x(x) = \int_{-\infty}^\infty f(x,y) dy \\
      &= e^{-x}\\
      \\
      f(y|x) &= \frac{f(x,y)}{f_x(x)}\\
      &= \frac{e^{-y}}{e^{-x}} \text{ if } y > x\\
      &= \frac{0}{e^{-x}} \text{ if } y \leq x
    \end{align*}
  \end{field}
\end{note}


\begin{note}
  \begin{field}
    Let ($X,Y$) be given as $f(x,y)$. Then $X$ and $Y$ are independent if
  \end{field}
  \begin{field}
    Let ($X,Y$) be given as $f(x,y)$. Then $X$ and $Y$ are independent if there exist functions $g(x), h(y)$ such that $f(x,y) = g(x)h(y)$ (factorization - don't need to compute marginals )
  \end{field}
\end{note}

\begin{note}
  \begin{field}
    Let $X,Y$ be independent. Then $E(g(X)h(Y)) = $
  \end{field}
  \begin{field}
    $E(g(X)h(Y)) = (E(g(X)))(E(h(Y)))$

    example: $E(X^2Y) = E(X^2)E(Y)$
  \end{field}
\end{note}

\begin{note}
  \begin{field}
    $X,Y$ independent\\ $Z = X + Y$  \\ $M_Z(t) = $
  \end{field}
  \begin{field}
    $M_Z(t) = M_x(t)M_Y(t)$
  \end{field}
\end{note}

\begin{note}
  \begin{field}
    Method of pdf bivariate
  \end{field}
  \begin{field}
    $f_{u,v}(u,v) = f_{x,y}(h_1(u,v),h_2(u,v))|J|$\\
    Where $|J| = \begin{vmatrix}
      \frac{\partial  x}{\partial  u} & \frac{\partial  x }{\partial v }\\ \frac{\partial  y }{\partial u } & \frac{\partial  y }{\partial v }
    \end{vmatrix} = \frac{\partial  x }{\partial u } \frac{\partial  y}{\partial v } - \frac{\partial  y }{\partial u } \frac{\partial  x }{\partial v }$

    and $u = g_1(x,y), v = g_2(x,y)$ and $x = h_1(x,y), y = h_2(x,y)$
  \end{field}
\end{note}


\begin{note}
  \begin{field}
    $X,Y$ independent, $g(X)$ a function only of $X$ and $h(Y)$ a function only of $Y$. Then
  \end{field}
  \begin{field}
    $g(X)$ and $g(Y)$ are independent.
  \end{field}
\end{note}

\begin{note}
  \begin{field}
    Correlation
  \end{field}
  \begin{field}
    $\rho_{XY} = \frac{Cov(X,Y)}{\sigma_X\sigma_Y}$
  \end{field}
\end{note}

\begin{note}
  \begin{field}
    $m$ independent trials, each trial resulting in one of $n$ outcomes, with probabilities $p_1, \ldots, p_n$. $X_i$ is the count of the number of times the $i$th outcome occured in the $m$ trials.
  \end{field}
  \begin{field}
    Multinomial distribution
    $f(x_1, \ldots, x_n) = \frac{m!}{x_1! \cdots x_n!}p_1^{x_i} \cdots p_n^{x_n}$
  \end{field}
\end{note}

\begin{note}
  \begin{field}
    $|E(XY)| \leq $ (Cauchy-Schwartz)
  \end{field}
  \begin{field}
    $|E(XY)| \leq E(|XY|) \leq (E(|X|^2))^{1/2}(E(|Y|^2))^{1/2}$
  \end{field}
\end{note}


\begin{note}
  \begin{field}
    $E(g(X)) \geq $ where $g$ is a convex function
  \end{field}
  \begin{field}
    $E(g(X)) \geq g(E(X))$ where $g$ is a convex function (Jensen's inequlity)
  \end{field}
\end{note}

\begin{note}
  \begin{field}
    Ranking of types of means
  \end{field}
  \begin{field}
    $\mu_{\text{harmonic}} \leq \mu_{\text{geometric}} \leq \mu_{\text{arithmetic}}$ By Jensens inequality (using logs )
  \end{field}
\end{note}

\begin{note}
  \begin{field}
    Linear transformations of multivariate normal $X \sim N(\vec{\mu},\Sigma) $\\
    $A \vec{X} + \vec{b}$
  \end{field}
  \begin{field}
    $A \vec{X} + \vec{b} \sim N(A \vec{\mu} + \vec{v}, A \Sigma A^t)$
  \end{field}
\end{note}

\begin{note}
  \begin{field}
    $X \sim N(\vec{\mu},\Sigma) $\\

    $\vec{X}_a | \vec{X}_b \sim $
  \end{field}
  \begin{field}
    $\vec{X}_a | \vec{X}_b \sim N\big(\vec{\mu_a} + \Sigma_{ab} \Sigma^{-1}_{bb}(\vec{x}_b - \vec{\mu}_b), \Sigma_{ba} - \Sigma_{ab}\Sigma_{bb}^{-1}\Sigma_{ba}$\\

ex: $(X_1,X_2,X_3), \vec{\mu} = (1,2,3)^t, \Sigma = \begin{pmatrix}
  3 & 1 & 0 \\ 1 & 3 & 1 \\ 0 & 1 & 3
\end{pmatrix}$
$X1,X_3 | X_2 = 1$\\
$a = \{1,3\}, b = \{2\}$\\
$\mu_a = (1,3)^t, \mu_b = 1$\\
$\Sigma_{aa} = \begin{pmatrix}
  3 & 0 \\ 0 & 3
\end{pmatrix}, \Sigma_{ab} = (1,1)^t$

  \end{field}
\end{note}

\begin{note}
  \begin{field}
    $(X,Y)$ multinomial \\
    $aX + bY \sim $
  \end{field}
  \begin{field}
    $aX + bY \sim N(a \mu_x + b\mu_y, a^2 \sigma_x^2 + b^2 \sigma_y^2 + 2 ab \rho \sigma_x \sigma_y)$
  \end{field}
\end{note}

\begin{note}
  \begin{field}
    $(X,Y)$ multinomial \\
    $Y|X\sim $
  \end{field}
  \begin{field}
    $Y|X\sim  N(\mu_y + \rho \frac{\sigma_y}{\sigma_x}(x - \mu_x), \sigma_Y^2(1 - \rho^2))$
  \end{field}
\end{note}

\begin{note}
  \begin{field}
    CDF for Max order statistic
  \end{field}
  \begin{field}
    $(F(x))^n$
  \end{field}
\end{note}

\begin{note}
  \begin{field}
    PDF for Max order statistic
  \end{field}
  \begin{field}
    $n(F(x))^{n-1}f(x)$
  \end{field}
\end{note}

\begin{note}
  \begin{field}
    CDF for Min order statistic
  \end{field}
  \begin{field}
    $1 - (1 - F(x))^n$
  \end{field}
\end{note}

\begin{note}
  \begin{field}
    PDF for Min order statistic
  \end{field}
  \begin{field}
    $n(1 - F(x))^{n-1}f(x)$
  \end{field}
\end{note}

\begin{note}
  \begin{field}
    CDF for $k$th order statistic
  \end{field}
  \begin{field}
    $F_{(k)}(x) = \sum_{j = k}^n \binom{n}{j}(F(x))^j(1 -F(x))^{n-j}$
  \end{field}
\end{note}

\begin{note}
  \begin{field}
    PDF for $k$th order statistic
  \end{field}
  \begin{field}
    $f_{(k)}(x) = k \binom{n}{k}f(x)F(x)^{k-1}(1 - F(x))^{n-k}$
  \end{field}
\end{note}

%%end_tag

%%end_tag


\end{document}
