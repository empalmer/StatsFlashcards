\documentclass[12pt]{article}
\usepackage[utf8]{inputenc}
\usepackage{amsmath,amssymb,amsthm}
\newenvironment{note}{\paragraph{NOTE:}}{}
\newenvironment{field}{\paragraph{field:}}{}
\newenvironment{tags}{\paragraph{tags:}}{}

\begin{document}


%% some stuff here too

\tags{Theory2}

%% some stuff
\begin{note}
  \begin{field}
    Definition of Convergence
  \end{field}
  \begin{field}
    A sequence $\{a_n\}_{n > 1}$ of real numbers is said to \textbf{converge} to a point $a \in \mathbb{R}$ if for any $\epsilon > 0$ there exists $N \in \mathbb{N}$ such that for all $m > N $ we have $|a_m - a| < \epsilon $
  \end{field}
\end{note}

\begin{note}
  \begin{field}
    Example of convergence: $a_n = \frac{1}{n}$
  \end{field}
  \begin{field}
    For any $\epsilon > 0$, choose $N$ such that $\frac{1}{N} < \epsilon$. Then for any $m > N$ we have that

    $$a_n = \frac{1}{n} < \frac{1}{N} < \epsilon$$

    and therefore $|a_m - 0| = \frac{1}{n} < \epsilon$
  \end{field}
\end{note}


\begin{note}
  \begin{field}
    Given two convergent sequences $\{a_n\}$ and $\{b_m\}$ such that $a_m \to a$ and $b_m \to b$\\
    $\lim_{n \to \infty} a_nb_n = $
  \end{field}
  \begin{field}
    Given two convergent sequences $\{a_n\}$ and $\{b_m\}$ such that $a_m \to a$ and $b_m \to b$\\
    $\lim_{n \to \infty} a_nb_n = (\lim_{n \to \infty}a_n)(\lim_{n \to \infty}b_n) = ab$
  \end{field}
\end{note}


\begin{note}
  \begin{field}
    Definition: Convergence in probability
  \end{field}
  \begin{field}
    A sequence of random variables $\{X_n\}_{n \geq 1}$ \textbf{converges in probability } to a random variable $X$, if for every $\epsilon > 0$, $$\lim_{n \to \infty}P(|X_n - X| \geq \epsilon) = 0 $$
    We write $X_n \overset{p}{\to} X$

    Equivalently, $x_m \overset{p}{\to} x$ if $\lim_{n \to \infty}P(|x_n - x| < \epsilon) = 1$
  \end{field}
\end{note}

\begin{note}
  \begin{field}
    Convergence in probability example:
    Let $\{x_n\}$ be a sequence of random variables such that $x_n \sim N(0,1/m^2)$\\
    Show that $x_n \overset{p}{\to} 0$:
  \end{field}
  \begin{field}
    Let $\epsilon > 0$. We obtain $P(|x_n - 0|) = P(x_n > \epsilon) + P(X_n < -\epsilon )$. ie we are looking at the tail probabilities.

    Now,
    \begin{align*}
      P(X_n < -\epsilon) + P(x_n > \epsilon) &=
      P(nx_n < n\epsilon) + P(nx_n > n\epsilon)\\
      &= \Phi(n\epsilon) + 1 - \Phi(n \epsilon)\\
      &= 2\Phi(-n\epsilon) \underset{n\to \infty}{\to}  0
    \end{align*}

    Therefore $x_n \overset{p}{\to} 0$
  \end{field}
\end{note}


\begin{note}
  \begin{field}
    Example convergence in probability
    Let $W \sim N(0,1)$ and $U \sim Unif(0,1)$, and define the sequence $\{x_n\}_{n \geq 1}$ as $x_n = W $with prob $1 - 1/n, U$ with prob $ 1/n$

    Show that $x_n \overset{p}{\to} W $

  \end{field}
  \begin{field}
    Let $\epsilon > 0$ Then.

    \begin{align*}
      P(|X_n - W| > \epsilon) &= P(|X_n - W| > \epsilon | X_n = W)P(X_n = W) \\
      &+ P(|X_n - W| > \epsilon | X_n = U)P(X_n = U)\\
      &= 0 \cdot (1 - 1/n) + p_n (1/n)
    \end{align*}

    Where $p_n  $ is a probability, and therefore $0 \leq p_n \leq 1 $

    It follows that $p_n \frac{1}{n} \underset{n \to \infty}{\to} 0$, and therefore $P(|X_n - W| > \epsilon ) \underset{n \to \infty}{\to} 0$, for all $\epsilon > 0$, so that $X_n \overset{p}{\to} W$.
  \end{field}
\end{note}

\begin{note}
  \begin{field}
    Does $X_n \overset{p}{\to} c$ imply $E(X_n) \to c $?
  \end{field}
  \begin{field}
    Let $X_n = 0$ with probability $1 - 1/n$, $n^2 $ with probability $1/n$
    Then $P(|X_n - 0| > \epsilon) \leq P(X_n = n^2) = 1/n \underset{n \to \infty}{\to} 0$
    On the other hand,
    $E(X_n) = 0 \cdot P(X_n =0) + n^2P(X_n = n^2) = 0 + n^2 \frac{1}{n} = n \underset{n \to \infty}{\to} \infty$.
    Therefore $X_n \overset{p}{\to} c$ does not imply $E(X_n) \to c $
  \end{field}
\end{note}

\begin{note}
  \begin{field}
    Does $E(X_n) \to c $ imply $X_n \overset{p}{\to}c$?
  \end{field}
  \begin{field}
    Let $X_n = 0, $ with prob $1 - 1/n$, $n$ with prob $1/n$.
    Then $E(X_n) = 0 \cdot P(X_n = 0) + n P(X_n = n) = 0  + n 1/n = 1$ for all n.
    But $P(|X_n - 0| > \epsilon) \leq P(X_n = n) = \frac{1}{n } \underset{n \to \infty}{\to} 0$
    It follows, $X_n \overset{p}{\to} 0 $, and therefore we have $E(X_n) \to c $does not imply $X_n \overset{p}{\to}c$
  \end{field}
\end{note}

\begin{note}
  \begin{field}
     Suppose $\{X_n\}_{n \geq 1}$ and $\{Y_n\}_{n \geq 1}$ be two sequences of random variables such that $X_n \overset{p}{\to} x_0$ and $Y_n \overset{p}{\to} y_0$ as $n \to \infty$, where $x_o, y_0 \in \mathbb{R}$

     What properties do we have?
  \end{field}
  \begin{field}
    \begin{itemize}
      \item $X_n \pm Y_m \overset{p}{\to} x_0 \pm y_0$ as $n$ increases to $\infty$
      \item $X_nY_n \overset{p}{\to} x_0y_0$ as $n$ increases to $\infty$
      \item $X_n/Y_n \overset{p}{\to} x_0/y_0$ as $n$ increases to infinity, provided that $P(Y_n = 0) = 0 $ fro all $n$ and $y_0 \neq 0$
    \end{itemize}
  \end{field}
\end{note}

\begin{note}
  \begin{field}
    Let $\{X_n\}_{n \geq 1}$ be a sequence of random variables such that $x_n \overset{p}{\to} x_0 \in \mathbb{R}$, as $n \to \infty$, and let $g: \mathbb{R} \to \mathbb{R}$ be a continuous function . Then $$g(X_n) \overset{p}{\to}  \text{ as } n \to \infty$$
  \end{field}
  \begin{field}
    Let $\{X_n\}_{n \geq 1}$ be a sequence of random variables such that $x_n \overset{p}{\to} x_0 \in \mathbb{R}$, as $n \to \infty$, and let $g: \mathbb{R} \to \mathbb{R}$ be a continuous function . Then $$g(X_n) \overset{p}{\to} g(x_0) \text{ as } n \to \infty$$
  \end{field}
\end{note}


\begin{note}
  \begin{field}
    Proof of: Let $\{X_n\}_{n \geq 1}$ be a sequence of random variables such that $x_n \overset{p}{\to} x_0 \in \mathbb{R}$, as $n \to \infty$, and let $g: \mathbb{R} \to \mathbb{R}$ be a continuous function . Then $$g(X_n) \overset{p}{\to} g(x_0) \text{ as } n \to \infty$$
  \end{field}
  \begin{field}
    Since $g$ is continuous at $X = x_0$, we have that for any $\epsilon > 0$, there exits $\delta > 0$ such that $|g(x) - g(x_0)| > \epsilon $ implies $|x - x_0| > \delta $

    We obtain

    $$0 \leq P(|g(X_n) - g(x_0)| > \epsilon ) \leq P(|X_n - x_0| > \delta) \underset{n \to \infty}{\to} 0$$
  \end{field}
\end{note}


\begin{note}
  \begin{field}
    Weak Law of Large numbers
  \end{field}
  \begin{field}
    Let $X_1, X_2, X_3 \ldots$ Be a sequence of iid random variables with $E(X_1) = \mu$ (finite) and $V(X_1) = \sigma^2 < \infty$, and define $\bar{X_n} = \frac{1}{n} \sum_{i = 1}^n X_i$ (the sample mean).

    Then $$ \bar{X_n }\overset{p}{\to}\mu \text{ as } n \to \infty$$

  \end{field}
\end{note}


\begin{note}
  \begin{field}
    Proof of Weak Law of Large Numbers
  \end{field}
  \begin{field}
    \begin{align*}
      P(|\bar{X_n} - \mu| > \epsilon ) &= P((\bar{X_n} - \mu)^2 > \epsilon^2)\\
      &\leq \frac{E((\bar{X_n} - \mu)^2)}{\epsilon^2} \text{ by Chebyshev's Inequality}\\
      &= \frac{V(\bar{X_n})}{\epsilon^2} \text{ by def of variance}\\
      &= \frac{\sigma^2}{n \epsilon^2} \underset{n \to \infty}{\to} 0
    \end{align*}

    Therefore $\bar{X_n} \overset{p}{\to} \mu$
  \end{field}
\end{note}

%%end_tag
%%start_tag Wk2

\begin{note}
  \begin{field}
    Consistency
  \end{field}
  \begin{field}
    If our estimate converges in probability to the value of the parameter of interest as the sample size $n$ increases
  \end{field}
\end{note}

\begin{note}
  \begin{field}
    Consistency of $S^2$
  \end{field}
  \begin{field}
    Suppose $X_1, X_2, \ldots$ is a sequence of iid random varialbes with $E(X_1) = \mu$ finite and $V(X_1) = \sigma^2 < \infty$

    and define $$ S_n^2 = \frac{1}{n-1}\sum_{i = 1}^n(X_i - \bar{X_n})^2 \quad \text{The sample variance}$$

    Can we show that $S_n^2$ is a consistent estimate of $\sigma^2$? In other words, can we show taht $S_n^2 \overset{p}{\to} \sigma^2 \text{ as } n \to \infty$

    Using Chebychev's inequality, we obtain

    \begin{align*}
      P(|S_n^2 - \sigma^2| > \epsilon ) &\leq \frac{E[(S_n^2 - \sigma^2)^2]}{\epsilon^2}\\
      &= \frac{V(S_n^2)}{\epsilon^2}
    \end{align*}

    There fore, a sufficient condition that $S_n^2$ converges in probablility to $\sigma^2$ is that the variance of $S_n^2$  $V(S_n^2) \to 0$, as $n \to \infty$
  \end{field}
\end{note}

\begin{note}
  \begin{field}
    $V(S_n^2) \to 0$ as long as
  \end{field}
  \begin{field}
    $V(S_n^2) \to 0$ as long as the fourth central moment $\mu_4 = E[(X_1 - \mu)^4]$ is finite.
  \end{field}
\end{note}

\begin{note}
  \begin{field}
    Khinchin's WLLN
  \end{field}
  \begin{field}
    Let $X_1, X_2, \ldots$ be a sequence of iid random variables with $E(X_1) = \mu$ (finite). Then, $\bar{X_n} \overset{p}{\to} \mu$ as $n \to \infty$
  \end{field}
\end{note}


\begin{note}
    \begin{field}
      Let $X_1, X_2 \ldots$ be a sequence of random variables, such that for some $r > 0$ and $c \in \mathbb{R}$, $E[|X_n - c|^r] \underset{n \to \infty}{\to} 0$. Then $X_n \overset{p}{\to} $, as $n \to \infty$
    \end{field}
    \begin{field}
      (A general result to establish convergence in probability )

      Let $X_1, X_2 \ldots$ be a sequence of random variables, such that for some $r > 0$ and $c \in \mathbb{R}$, $E[|X_n - c|^r] \underset{n \to \infty}{\to} 0$. Then $X_n \overset{p}{\to} c$, as $n \to \infty$
    \end{field}
\end{note}

\begin{note}
  \begin{field}
    Consistent estimator for $X_1, X_2, \ldots X_n \sim $ iid Univorm($0,\theta$), $\theta > 0$. ( and sketch of proof )
  \end{field}
  \begin{field}
    $X_{(n)} = \max(X_1, \ldots X_n)$ (the largest order statistic)

    Proof

    First recall that the pdf of $X_{(n)}$ is given by $$f(x) = nx^{n-1} \theta^{-n}, 0 < x < \theta, 0 \text{otherwise} $$

    We obtain

    \begin{align*}
      E(X_{(n)}) &= \int_{0}^\theta x f(x) dx \\
      &= n \theta^{-n} \int_0^\theta x^n dx \\
      &= \frac{n}{n-1}\theta\\
      E(X_{(n)}^2) &= \int_0^\theta x^2f(x)dx \\
      &= n\theta^{-n} \int_0^\theta x^{n+1}dx \\
      &= \frac{n}{n + 2} \theta^2
    \end{align*}

    We have

    \begin{align*}
      E[(X_{(n)} - \theta)^2] &= E(X_{(n)}^2) - 2\theta E(X_{(n)}) + \theta^2 \\
      &= \frac{n}{n + 2}\theta^2 - 2\theta \frac{n}{n + 1} \theta + \theta^2\\
      &\cdots\\
      &= \frac{2\theta^2}{(n + 1)(n + 2)} \underset{n \to \infty}{\to} 0
    \end{align*}

    Hence, taking $c = 0$ and $r = 2$, from the previous theorem, we obtain $X_{(n)} \overset{p}{\to} \theta$ as $n \to \infty$

  \end{field}
\end{note}



\begin{note}
  \begin{field}
    Definition Almost Sure Convergence
  \end{field}
  \begin{field}
    A sequence $\{X_n\}_{n \geq 1}$ of random variables is said to converge \textbf{Almost Surely} to a random variable $X$ if for every $\epsilon > 0$,
    $$ P(\lim_{n \to \infty}|X_n - X| > \epsilon) = 0$$

    We write $X_n \overset{a.s}{\to} X$ as $n \to \infty$

  \end{field}
\end{note}



\begin{note}
  \begin{field}
    Strong Law of Large Numbers
  \end{field}
  \begin{field}
    Let $X_1, X_2, \ldots$ be an iid sequence of random variables, with $E(X_1) = \mu$ (finite) and $V(X_1) = \sigma^2 < \infty$. Then,
    $$\bar{X_n} \overset{a.s}{\to} \mu \quad \text{as } \mu \to \infty$$
  \end{field}
\end{note}

\begin{note}
  \begin{field}
    Does convergence in probability imply convergence almost surely?
  \end{field}
  \begin{field}
    No.
    Let $\Omega = [0.1]$, with uniform probability distribution.  Define the sequence $\{X_n\}_{n \geq 1}$ as:
    \begin{align*}
      X_1(\omega) &= \omega + \mathbb{I}_{[0,1]}(\omega)\\
      X_2(\omega) &= \omega + \mathbb{I}_{0,1/2}(\omega)\\
      X_3(\omega) &= \omega + \mathbb{I}_{1/2,1}(\omega)\\
      X_4(\omega) &= \omega + \mathbb{I}_{0,1/3}(\omega)\\
      X_5(\omega) &= \omega + \mathbb{I}_{1/3,2/3}(\omega)\\
      &\vdots
    \end{align*}
    $X_5(\omega) = \omega + 1$

    Let $X(\omega) = \omega$, then it is easy to show that $X_n \overset{p}{\to} X$ because $P(|X_n - X| \geq \epsilon) = P([a_n,b_n])$, where $l_n = \text{length}([a_n, b_n]) \underset{n \to \infty}{\to} 0$.

    However $X_n$ does not converge to $X$ almost surely, because for every $\omega \in [0,1]$, alternates between $\omega$ and $\omega + 1$, infinetly often as $n \to \infty$
  \end{field}
\end{note}



\begin{note}
  \begin{field}
    Convergence in Distribution
  \end{field}
  \begin{field}
    A sequence $\{X_n\}_{n \geq 1}$ of random variables converges in distribution to a random variable $X$ if,
    $$ \lim_{n \to \infty} F_{X_n}(x) = F_X(x)$$

    at all points $x$ where $F_X(x)$ is continuous

    We write $X_n \overset{d}{\to} X$
  \end{field}
\end{note}

\begin{note}
  \begin{field}
    Example of convergence in distribution

    Let $X_n \sim N(0, \frac{n+1}{n})$, and $X \sim N(0,1)$. We want to show that $X_n \overset{d}{\to} X$.
  \end{field}
  \begin{field}
    \begin{align*}
      P(X_n \leq X) &= P(\sqrt{\frac{n}{n + 1}}X_n \leq \sqrt{\frac{n}{n+1}}x)\\
      &= \Phi(\sqrt{\frac{n}{n + 1}}x) \underset{n \to \infty}{\to} \Phi(x)
    \end{align*}

    And we obtain that $F_{X_n} \to \Phi(x) = F_X(x), \forall x$, and therefore $X_n \overset{d}{\to} X$
  \end{field}
\end{note}

%%end_tag
%%start_tag Wk3

\begin{note}
  \begin{field}
    Does Convergence in probability imply convergence in distribtuion?
  \end{field}
  \begin{field}
    Yes
  \end{field}
\end{note}

\begin{note}
  \begin{field}
    Does Convergence in distribution imply convergence in probability?
  \end{field}
  \begin{field}
    No - unless converges in distribution to a constant
  \end{field}
\end{note}

\begin{note}
  \begin{field}
    A sequence $\{X_n\}_{n \geq 1}$ of random variables converges in probability to a constant $c \in \mathbb{R}$ if and only if
  \end{field}
  \begin{field}
    A sequence $\{X_n\}_{n \geq 1}$ of random variables converges in probability to a constant $c \in \mathbb{R}$ if and only if the sequence converges in distribution to $c$
  \end{field}
\end{note}

\begin{note}
  \begin{field}
    If $X_n \overset{d}{\to} X$ and $Y_n \overset{d}{\to} Y$ we have that
    \begin{enumerate}
      \item $X_n \pm Y_n $
      \item $X_nY_n $
    \end{enumerate}
  \end{field}
  \begin{field}
    In general it is not true that if $X_n \overset{d}{\to} X$ and $Y_n \overset{d}{\to} Y$ we have that
    \begin{enumerate}
      \item $X_n \pm Y_n \overset{d}{\to} X + Y$
      \item $X_nY_n \overset{d}{\to} XY$
    \end{enumerate}
  \end{field}
\end{note}

\begin{note}
  \begin{field}
    Let $\{X_n\}_{n \geq 1}$ be a sequence of random variables such that $X_n \overset{d}{\to}X$, for some random variable $X$ (possibly a constant). Then for any continuous funciton $g:\mathbb{R} \to \mathbb{R}$, we have $g(X_n) \overset{d}{\to} $
  \end{field}
  \begin{field}
    Let $\{X_n\}_{n \geq 1}$ be a sequence of random variables such that $X_n \overset{d}{\to}X$, for some random variable $X$ (possibly a constant). Then for any continuous funciton $g:\mathbb{R} \to \mathbb{R}$, we have $g(X_n) \overset{d}{\to} g(X)$
  \end{field}
\end{note}

\begin{note}
  \begin{field}
    Let $\{X_n\}_{n \geq 1}$ and $\{Y_n\}_{n \geq 1}$ be two sequences of random variables such that $X_n \overset{d}{\to}X$ for some random variable $X$ (possibly a constant) and $Y_n \overset{p}{\to} c \in \mathbb{R}$

    Then, as $n \to \infty$,

    \begin{enumerate}
      \item $X_n \pm Y_n \overset{d}{\to} $
      \item $X_nY_n \overset{d}{\to} $
      \item $X_n/Y_n \overset{d}{\to} \quad \text{ provided } P(Y_n = 0) = 0 \forall n \text{ and } c \neq 0$
    \end{enumerate}
  \end{field}
  \begin{field}
    Slutsky's Theorem
    Let $\{X_n\}_{n \geq 1}$ and $\{Y_n\}_{n \geq 1}$ be two sequences of random variables such that $X_n \overset{d}{\to}X$ for some random variable $X$ (possibly a constant) and $Y_n \overset{p}{\to} c \in \mathbb{R}$

    Then, as $n \to \infty$,

    \begin{enumerate}
      \item $X_n \pm Y_n \overset{d}{\to} X \pm c$
      \item $X_nY_n \overset{d}{\to} cX$
      \item $X_n/Y_n \overset{d}{\to} X/c \quad \text{ provided } P(Y_n = 0) = 0 \forall n \text{ and } c \neq 0$
    \end{enumerate}
  \end{field}
\end{note}

\begin{note}
  \begin{field}
    Central Limit Theorem
  \end{field}
  \begin{field}
    Let $X_1, X_2, \ldots$ be an iid sequence of random variables, with $E(X_1) = \mu$(finite) and $V(X_1) = \mu^2 < \infty$

    Then, for $\bar{X_n} = \frac{1}{n }\sum_{i = 1}^\infty X_i$ (the sample mean), we have that
    $$ \frac{\sqrt{n}(\bar{X_n} - \mu)}{\sigma} \overset{d}{\to} N(0,1) \quad \text{as } n \to
    \infty$$
  \end{field}
\end{note}

\begin{note}
  \begin{field}
    Equivalent results of CLT
  \end{field}
  \begin{field}
    \begin{itemize}
      \item $\frac{(\bar{X_n} - \mu)}{\frac{\sigma}{\sqrt{n}}} \overset{d}{\to}N(0,1)$
      \item $\sqrt{n}(\bar{X_n} - \mu) \overset{d}{\to} N(0,\sigma^2)$
      \item $\frac{\sum_{i=1}^n X_i - n\mu}{\sqrt{n}\sigma} \overset{d}{\to}N(0,1)$
      \item $\bar{X_n} \overset{d}{\to} N(\mu, \sigma^2/n)$
    \end{itemize}
  \end{field}
\end{note}


\begin{note}
  \begin{field}
    Let $\{X_n\}_{n \geq 1}$ be a sequence of random variables such that the mgf $M_{X_n}(t)$ of $X_n$ exists in a neighborhood of 0, for all , and suppose that

    $$ \lim_{n \to \infty} M_{X_n}(t) = M_X(t) \quad \text{for all } t \text{ in a neighborhood of }0$$
    where $M_X(t)$ is the mgf for some random variable $X$. Then,
  \end{field}
  \begin{field}
    Let $\{X_n\}_{n \geq 1}$ be a sequence of random variables such that the mgf $M_{X_n}(t)$ of $X_n$ exists in a neighborhood of 0, for all , and suppose that

    $$ \lim_{n \to \infty} M_{X_n}(t) = M_X(t) \quad \text{for all } t \text{ in a neighborhood of }0$$
    where $M_X(t)$ is the mgf for some random variable $X$. Then, there exists a unique cdf $F_x(x)$ whose moments are determined by $M_y(t)$ and for all $x$, where $F_x(x)$ is continuous we have $\lim_{n \to \infty} F_{X_n}(x) = F_x(x)$
  \end{field}
\end{note}
%%end_tag
%%start_tag Wk4

\begin{note}
  \begin{field}
    $\frac{\sqrt{n}(\bar{X} - \mu)}{S_n} \overset{d}{\to} $
  \end{field}
  \begin{field}
    Using the CLT, and slutsky's theorem, we have

    $$\frac{\sqrt{n}(\bar{X} - \mu)}{S_n} = \frac{\sigma}{S_n}\cdot \frac{\sqrt{n}(\bar{X} - \mu)}{\sigma} \overset{d}{\to} N(0,1) $$
  \end{field}
\end{note}

\begin{note}
  \begin{field}
    $ g(X) \approx $ \\
$E(g(X)) \approx $, $V(g(X))\approx $
  \end{field}
  \begin{field}
    $$ g(X) \approx g(\mu) + g'(X)(X-\mu)$$ Using a first order taylor approximation
    $E(g(X)) \approx g(\mu)$, $V(g(X))\approx [g'(\mu)]^2V(X)$
  \end{field}
\end{note}

\begin{note}
  \begin{field}
    Delta Method
  \end{field}
  \begin{field}
    Let $\{Y_n\}_{n \geq 1}$ be a sequence of random variables such that $\sqrt{n}(Y_n - \theta) \overset{d}{\to} N(0,\sigma^2)$ as $n \to \infty$. Suppose that for a given function $g$ and a specific value of $\theta$, $g'(\theta)$ exists and is not equal to zero. Then
    $$\sqrt{n}(g(Y_n) - g(\theta)) \overset{d}{\to} N(0,\sigma^2[g'(\theta)]^2)$$

    as $n\to \infty$
  \end{field}
\end{note}

\begin{note}
  \begin{field}
    Second Order delta method
  \end{field}
  \begin{field}
    Let $\{Y_n\}_{n \geq 1}$ be a sequence of random variables such that $\sqrt{n}(Y_n - \theta ) \overset{d}{\to} N(0,\sigma^2)$ as $n\to \infty$.
    And that for a given function $g$ as specific value of $\theta$, we hvae $g'(\theta)=0$, but $g''(\theta)$ Exists and is not equal to 0. Then

    $$\sqrt{n}(g(Y_n) - g(\theta)) \overset{d}{\to} \sigma^2 \frac{g''(\theta)}{2}\chi^2_1 \quad \text{as } n \to \infty$$

  \end{field}
\end{note}

\begin{note}
  \begin{field}
    $\chi_n^2 \dot{\sim} $ for sufficiently large $n$
  \end{field}
  \begin{field}
    $\chi_n^2 \dot{\sim} N(n,2n)$
  \end{field}
\end{note}

\begin{note}
  \begin{field}
    Definition Statistic
  \end{field}
  \begin{field}
    Let $X_1, \ldots, X_n$ be a random sample from a given population. Then, any \underline{observable} real-valued (or vector-valued) function $T(\mathbf{X}) = T(X_1, \ldots, X_n)$ of the random variables $X_1, \ldots, X_n$ is called a \textbf{Statistic}
  \end{field}
\end{note}

\begin{note}
  \begin{field}
    Sampling Distribution
  \end{field}
  \begin{field}
    The probability distribution of the statitic $T(\mathbf{X})$ is called the \textbf{Sampling Distribution} of $T(\mathbf{X})$
  \end{field}
\end{note}


\begin{note}
  \begin{field}
    Sufficient Statistic
  \end{field}
  \begin{field}
    A statistic $T(\mathbf{X})$ is a \textbf{Sufficient Statistic} for $\theta$, if the conditional distribution of the sample $\mathbf{X}$ given the value of $T(\mathbf{X})$ does not depend on $\theta$
  \end{field}
\end{note}
%%end_tag
%%start_tag Wk5

\begin{note}
  \begin{field}
    Determine if $T(\mathbf{X}) = \sum X_i$ where $X_i\sim Bern(p)$ is sufficient for $p$ using definition of sufficiency
  \end{field}
  \begin{field}
    \begin{align*}
      P(\mathbf{X} = \mathbf{x}\big| T = t) &= \frac{P(\cap_{i = 1}^n X_i = x_i)}{P(T = t)}\\
      &= \prod_{i = 1}^n \frac{P(X_i = x_i)}{P(T = t)} \quad \text{by independence}\\
      &= \frac{p^{\sum_{i = 1}^n x_i}(1 - p)^{n - \sum_{i = 1}^n x_i}}{\binom{n}{t}p^t(1 -p)^{n-t}} \quad \text{Because } T \sim \text{Binom}(n,p)\\
      &= \frac{p^t(1-p)^{n-t}}{\binom{n}{t}p^t(1-p)^{n-t}} \quad \text{because }t = \sum_{i = 1}^n x_i\\
      &= \frac{1}{\binom{n}{t}} \quad \text{which is free of  }p
    \end{align*}
  \end{field}
\end{note}

\begin{note}
  \begin{field}
    How to show sufficiency (not using factorization)
  \end{field}
  \begin{field}
    Let $p(\mathbf{X}|\theta)$ be the joint PDF or PMF of $\mathbf{X}$ and $q(t|\theta)$ the PDF or PMF of the statistic $T(\mathbf{X})$. Then $T(\mathbf{X})$ is a sufficient statistic for $\theta$ if for every $\mathbf{X}$ in the sample space, the ratio

    $$ \frac{p(\mathbf{x}|\theta)}{q(T(\mathbf{x})|\theta)}$$

    is constant as a function of $\theta$
  \end{field}
\end{note}

\begin{note}
  \begin{field}
    Suppose that $X_1, \ldots X_n$ are iid $N(\mu,\sigma^2)$ where $\sigma^2$ is known. If the statistic $T(\mathbf{X}) = \bar{X_n}$ sufficient for $\mu$?
  \end{field}
  \begin{field}
    \begin{align*}
       \frac{f(\mathbf{x}|\mu)}{q(T(\mathbf{X})|\mu)} &=
       \frac{(2\pi\sigma^2)^{n/2}e^{-\frac{1}{2\sigma^2}[\sum_{i = 1}^n(x_i - \bar{x})^2 + n(\bar{x} - \mu)^2]}}{(2\pi\sigma/n)^{-1/2}e^{-\frac{1}{2\sigma^2}(\bar{x}-\mu)^2}}\\
       &= n^{-1/2}(2\pi\sigma^2)^{-(n-1)/2}e^{-\frac{1}{2\sigma^2}\sum_{i = 1}^n(x_i - \bar{x})^2}
    \end{align*}

    Which does not depend on $\mu$, and therefore $\bar{X_n}$ is suffient for $\mu$ as long as $\sigma^2$ is known
  \end{field}
\end{note}

\begin{note}
  \begin{field}
    The joint pdf of the sample $\mathbf{X} =(X_1, X_2, \ldots X_n)$ isSuppose that $X_1, \ldots X_n$ are iid $N(\mu,\sigma^2)$ where $\sigma^2$ is known.
  \end{field}
  \begin{field}
    \begin{align*}
      f(\mathbf{x}|\mu) &= \prod_{i = 1}^n \frac{1}{\sqrt{2\pi\sigma^2}} e^{\frac{-1}{2\sigma^2}(x_i - \mu)^2}\\
      &= (2\pi\sigma^2)^{n/2} e^{-1/2\sigma^2 \sum_{i = 1}^n (x_i - \mu)^2}\\
      &= (2\pi\sigma^2)^{n/2} e^{-1/2\sigma^2 \sum_{i = 1}^n (x_i - \bar{x} + \bar{x} - \mu)^2}\\
      &= (2\pi\sigma^2)^{n/2} e^{-1/2\sigma^2 \sum_{i = 1}^n (x_i - \bar{x})^2 + 2(\bar{x} - \mu)\sum_{i = 1}^n (x_i - \bar{x}) + n(\bar{x} - \mu)^2}\\
      &= (2\pi\sigma^2)^{n/2}e^{-1/2\sigma^2 (\sum_{i = 1}^n (x_i - \bar{x})^2 + n(\bar{x} - \mu)^2)}
    \end{align*}
  \end{field}
\end{note}

\begin{note}
  \begin{field}
    Show a statistic $T(\mathbf{X})$ is sufficient
  \end{field}
  \begin{field}
    Neyman factorization theorem
    Let $f(\mathbf{x}|\theta)$ denote the joint pdf or pmf of the sample $\mathbf{X}$, A statistic $T(\mathbf{X})$ is a sufficient statistic for $\theta$ if and only if there exists functions $g(t|\theta)$ and $h(\mathbf{x})$ such that for all sample points $\mathbf{x}$ and all values of $\theta$ we can write
    $$ f(\mathbf{x}|\theta) = g(T(\mathbf{}x)|\theta)h(\mathbf{x})$$

    Note, in the theorem
    \begin{itemize}
      \item The function $g(T(\mathbf{X})|\theta)$ depends on $\mathbf{x} = (x_1, \ldots x_n)$ only through the statistic $T(\mathbf{X})$.
      \item The function $h(\mathbf{X})$ does not depend on $\theta$
    \end{itemize}
  \end{field}
\end{note}

\begin{note}
  \begin{field}
    Exponential Family
  \end{field}
  \begin{field}
    $$f(\mathbf{X}|\bf{\theta}) = h(\mathbf{x})c(\mathbf{\theta})e^{\sum _{i = 1}^n w_i(\bf(\theta))t_i(x)}$$
  \end{field}
\end{note}



%%end_tag
%%start_tag Wk6

\begin{note}
  \begin{field}
    Sufficiency in the exponential family
  \end{field}
  \begin{field}
    Let $X_1, \ldots , X_n$ be iid observations from a PDF or PMF, $f(x|\boldsymbol\theta)$ that belongs to an exponential family of the form

    $$ f(x|\boldsymbol\theta) = h(x)c(\boldsymbol\theta)e^{\sum _{i = 1}^k w_i(\boldsymbol\theta)t_i(x)}$$

    Where $\boldsymbol\theta = (\theta_1, \ldots, \theta_d)$, $d\leq k$. Then
    $$ T(\mathbf{X}) = \big(\sum _{j = 1}^k t_i(x_j), \cdots , \sum _{j = 1}^k t_k(x_j)\big)$$
  \end{field}
\end{note}


\begin{note}
  \begin{field}
    Minimal Sufficient Statistic
  \end{field}
  \begin{field}
    A sufficient statistic $T(\mathbf{X})$ is called a \textbf{Minimal Sufficient Statistic} if for any other sufficient statistic $T'(\mathbf{X})$, $T(\mathbf{X})$ is a function of $T'(\mathbf{X})$
  \end{field}
\end{note}

\begin{note}
  \begin{field}
    Determining if a statistic is minimal sufficient
  \end{field}
  \begin{field}
    Let $f(x|\theta)$ be the PDF or PMF of a sample $\mathbf{X}$. Suppose there exists a function $T(x)$ such that, for every two sample points, $\mathbf{x}$ and $\mathbf{y}$, the ratio $\frac{f(\mathbf{x}|\theta)}{f(\mathbf{y}|\theta)}$ is constant as a fucntion of $\theta$ iff and only if $T(\mathbf{x}) = T(\mathbf{y})$. Then $T(\mathbf{x})$ is a minimal sufficient statistic for $\theta$.
  \end{field}
\end{note}


\begin{note}
  \begin{field}
    Example of finding a minimal sufficient statistic:
    Suppose that $X_1, \ldots , X_n$ are idd Bernoulli($p$). What is a minimal sufficient statistic for $p$?
  \end{field}
  \begin{field}
    \begin{align*}
      f(\mathbf{x}|p) &= \prod_{i = 1}^n p^{x_i}(1-p)^{1 - x_i}\\
      &= p^{\sum _{i = 1}^n x_i}(1-p)^{n - \sum _{i = 1}^nx_i}
    \end{align*}

    And therefore for any two sample points $\mathbf{x}$ and $\mathbf{y}$, we obtain

    \begin{align*}
      \frac{f(\mathbf{x}|p)}{f(\mathbf{y}|p)} &= \frac{p^{\sum _{i = 1}^n x_i}(1-p)^{n - \sum _{i = 1}^nx_i}}{p^{\sum _{i = 1}^n y_i}(1-p)^{n - \sum _{i = 1}^ny_i}}\\
      &= p^{\sum _{i = 1}^n x_i - \sum _{i = 1}^n y_i}(1-p)^{\sum _{i = 1}^n y_i - \sum _{i = 1}^nx_i}
    \end{align*}

    Which is constant as a function of $p$ iff $\sum _{i = 1}^n x_i = \sum _{i = 1}^n y_i$

    Hence it follows from Lehman-Sheffe that $T(\mathbf{x}) = \sum _{i = 1}^n x_i$ is minimal sufficient for $p$
  \end{field}
\end{note}

\begin{note}
  \begin{field}
    Minimal sufficient statistic for $\mu,\sigma^2$, where the Xs are $N(\mu,\sigma^2)$
  \end{field}
  \begin{field}
    $T(\mathbf{x}) = (\bar{x},S_x^2)$ by Lehmann-Schaffe is minimal sufficient.
  \end{field}
\end{note}

\begin{note}
  \begin{field}
    Facts about sufficiency
  \end{field}
  \begin{field}
    \begin{itemize}
      \item The entire sample $\mathbf{X}$ is always sufficeint.
      \item Any one-to-one funciton of a minimal sufficient statisitc is also a minimal sufficient statistic
    \end{itemize}
  \end{field}
\end{note}


\begin{note}
  \begin{field}
    Ancillary Statistic
  \end{field}
  \begin{field}
    A statistic $S(\mathbf{X})$ whose distribution does not depend on the parameter $\theta$ is called an ancillary statistic for $\theta$
  \end{field}
\end{note}



%%end_tag
% Maybe add some stuff about ancilary statistics
%%start_tag Wk7
\begin{note}
  \begin{field}
    Complete statistic
  \end{field}
  \begin{field}
    Let $f(t|\theta)$ be the family of pdf's or pmfs for a statistic $T= T(\mathbf{x})$.

    The family of probability distributions is called \textbf{complete} (with respect to $\theta$) if $E_\theta(g(t)) = 0$ for all $\theta$, implies $P_\theta(g(T) = 0) = 1$ for all $\theta$

    Equivalently, we say that $T = T(\mathbf{X})$ is a complete statistic.

    In short, a statistic $T = T(\mathbf{x})$ is complete, if $E_\theta(g(T)) = 0$ for all $\theta$ implies $g(t) = 0$ with probability 1
  \end{field}
\end{note}

\begin{note}
  \begin{field}
    (Binomial complete sufficient statistic)
  \end{field}
  \begin{field}
    Suppose the statistic $T \sim Binom(n,p)$, $0 < p < 1$, and let $g$ be a function such that $E_p(g(T)) = 0$ for all $p$.

    Then, with $r = (\frac{p}{1-p})^t$

    \begin{align*}
      0 &= E_p(g(T))\\
      &= \sum_{t=0}^n g(t) \binom{n}{t}p^t(1-p)^{n-1}\\
      &= (1-p)^n \sum_{t = 0}^n g(t)\binom{n}{t}(\frac{p}{1-p})^t\\
      &= (1-p)^n \sum_{t = 0}^n g(t)\binom{n}{t}r^t\\
      &= \neq 0 \cdot \text{This is a polynomial of degree }n \text{ in } r \text{ with coefficients } g(t)\binom{n}{t}
    \end{align*}

    For the polynomial to be $0$ for all $r$ (and consequently for all $p$) each coefficient must be zero and therefore it must be the case that $g(t) = 0$ for $t = 0, 1, 2, \cdots , n$ Since $T \sim Binom(n,p)$, we have that $T$ takes on the values $t = 0,1,2,\ldots n$ with probability $1$ and therefore, we obtain $P_p(g(T)=0)=1$. Hence T is a complete statistic.
  \end{field}
\end{note}


\begin{note}
  \begin{field}
    Uniform complete sufficient statistic
  \end{field}
  \begin{field}
    Suppose that $X_1, \ldots , X_n$ are iid Uniform($0,\theta$), $\theta > 0$. We know that $T(\mathbf{X}) = X_{(n)}$ (the max order statistic) is sufficient for $\theta$. Furtheremore ,
    $$ f(t|\theta) = nt^{n-1}\theta^{-n} \quad 0 < t < \theta$$
    Now suppose that $g(t)$ is a function satisfying $E_\theta(g(T)) =0, \forall \theta$
    Differentiating on both sides with respect to $\theta$,

    \begin{align*}
       0 &= \frac{d}{d\theta} E_\theta(g(t))\\
       &= \frac{d}{d\theta} \int_0^\theta g(t)nt^{n-1}\theta^{-n} dt\\
       &= \theta^{-n}\frac{d}{d\theta} \int_0^\theta g(t)nt^{n-1} dt + (\frac{d}{d\theta}\theta^{-n}) \int_0^\theta g(t)nt^{n-1}dt \\
       &= \theta^{-n}g(\theta)n\theta^{n-1} + 0
    \end{align*}

    Since $n\theta^{-1} \neq 0$, we must have that $g(\theta) =0 \quad \forall \theta> 0$. And therefore $T$ is complete.
  \end{field}
\end{note}

\begin{note}
  \begin{field}
    Does minimal sufficent imply complete?
  \end{field}
  \begin{field}
    No

    Suppose that $X_1, \ldots X_n$ are iid $N(\theta,\theta^2)$ where $\theta \in \mathbb{R}$ is the unknown parameter of interest.

    We have

    \begin{align*}
      \frac{f(\mathbf{x}|\theta)}{f(\mathbf{y}|\theta)}
      &= \frac{(2\phi\sigma^2)^{-n/2}e^{-\frac{1}{2\sigma^2}\sum_{i =1}^n (x_i -\theta)^2}}{(2\phi\sigma^2)^{-n/2}e^{-\frac{1}{2\sigma^2}\sum_{i =1}^n (y_i -\theta)^2}}\\
      &= \frac{e^{-\frac{1}{2\sigma^2}[\sum _{i = 1}^n x_i^2 - 2\theta \sum _{i = 1}^n x_i]}}{e^{-\frac{1}{2\sigma^2}[\sum _{i = 1}^n y_i^2 - 2\theta \sum _{i = 1}^n y_i]}}
    \end{align*}

    Which is free of $\theta$ if $\sum _{i = 1}^n x_i^2 = \sum _{i = 1}^n y_i^2$ and $\sum _{i = 1}^n x_i = \sum _{i = 1}^n y_i$

    It follows that $T(\mathbf{X}) = (\sum _{i = 1}^n x_i, \sum _{i = 1}^n x_i^2)$ is minimal sufficient for $\theta$

    Now observe that $T_1(\mathbf{X}) = \sum _{i = 1}^n x_i \sim N(n\theta, n\theta^2)$ and therefore

    \begin{align*}
      E(T_1^2) &= V(T_1) + [E(T_1)]^2\\
      &= n\theta^2 + n^2\theta^2\\
      &= n\theta^2(1 + n)
    \end{align*}

    On the other hand, for $T_2 = \sum _{i = 1}^n x_i^2$,

    \begin{align*}
      E(T_2) &= n E(X_1)^2\\
      &= n[V(X_1) + [E(X_1)]^2]\\
      &= n\theta^2 + n\theta^2\\
      &= 2n\theta^2
    \end{align*}

    Then, taking $h(t_1,t_2) = 2t_1^2 - (n+1)t_2$, we have
    \begin{align*}
      E_\theta[h(T_1,T_2)] &= E_\theta[2T_1^2 - (n+1)T_2]\\
      &= 2E_\theta(T_1^2) - (n+1)E(T_2)\\
      &= 2n(n+1)\theta^2 - 2n(n+1)\theta^2\\
      &= 0 \quad \forall \theta
    \end{align*}

    But because $h(\mathbf{t}) \neq 0 \quad \forall \theta$, we have that $T(\mathbf{X})$ is not complete.
  \end{field}
\end{note}

\begin{note}
  \begin{field}
    Complete statistics in the exponential family
  \end{field}
  \begin{field}
    Let $X_1, \ldots , X_n$ be iid observations from an exponential family. with PDF or PMF of the form
    $$ f(x|\theta) = h(x)c(\theta)e^{\sum_{j=1}^k \omega_j(\theta_j)t_j(x)}$$

    Where $\boldsymbol\theta = (\theta_1, \ldots , \theta_k)$

    Then, the statistic $T(\mathbf{X}) = (\sum_{i=1}^n t_1(x_i), \sum _{i = 1}^n t_2(x_i), \ldots , \sum _{i = 1}^n t_k(x_i))$ is complete, as long as the parameter space $\Theta$ contains an open set in $\mathbb{R}^k$
  \end{field}
\end{note}

\begin{note}
  \begin{field}
    Suppose that a statistic $T$ is complete and let $g$ be a one-to-one function. Is the statistic $U = g(T)$ also complete?
  \end{field}
  \begin{field}
    Yes
  \end{field}
\end{note}

\begin{note}
  \begin{field}
    Does complete statistic imply minimial sufficient statistic?
  \end{field}
  \begin{field}
    If a minimal sufficient statistic exists, then any complete statistic is also a minimal sufficient statistic
  \end{field}
\end{note}

\begin{note}
  \begin{field}
    Basu's Theorem
  \end{field}
  \begin{field}
    If $T(\mathbf{x})$ is a complete and minimal sufficient statistic, then $T(\mathbf{x})$ is an independent of every ancillary statistic.
  \end{field}
\end{note}

\begin{note}
  \begin{field}
    Likelihood function
  \end{field}
  \begin{field}
    Let $f(\mathbf{x}|\theta)$ denote the joint pdf or pmf of the sample $\mathbf{X} = (X_1, \ldots , X_n)$, then given that $\mathbf{X} = \mathbf{x}$ is observed, the function of $\theta$ defined as $$L(\theta|\mathbf{x}) = f(\mathbf{x}|\theta)$$
    is called the Likelihood Function
  \end{field}
\end{note}

\begin{note}
  \begin{field}
    Idea of likelihood function
  \end{field}
  \begin{field}
    Suppose that $\mathbf{X}$ is a discrete random vector (so we can interpret probabilities easier)

    Then $L(\theta|\mathbf{x}) = P_\theta(\mathbf{X} = \mathbf{x})$. Now if we compare the likelihood function at two parameter values $\theta_1, \theta_2$ and we observe that $$P_{\theta_1}(\mathbf{X} = \mathbf{x}) = L(\theta_1|\mathbf{x}) > L(\theta_2|\mathbf{x}) = P_{\theta_2}(\mathbf{X} = \mathbf{x}$$
    Then, the sample point $\mathbf{x}$ that we actually observed is more likely to have occured if $\theta = \theta_1$, than if $\theta= \theta_2$, which can be interpreted as that $\theta_1$, is a more plausible value for the true value of $\theta$ than $\theta_2$ is.
  \end{field}
\end{note}


\begin{note}
  \begin{field}
    Fisher information - one parameter case
  \end{field}
  \begin{field}
    Let $X$ be a random variable with pdf or pmf $f(x |\theta)$ where $\theta \in \Theta \subseteq \mathbb{R}$

    (Fisher ) information about $\theta$ contained in $X$ is

    $$ I_{X}(\theta) = E_\theta[ \big(\frac{\partial}{\partial \theta} \log f(x|\theta)\big)^2]$$
  \end{field}
\end{note}

\begin{note}
  \begin{field}
    Example of one parameter case Fisher information
    Suppose that $X \sim Bern(p)$ What is the information that $X$ contains about the parameter $p$?
  \end{field}
  \begin{field}
    We have that $f(x|p) = p^x(1-p)^{1-x}$. Then $$ \log f(x|p) = x \log p + (1-x)\log(1-p)$$
    $$ \frac{\partial}{\partial p} \log f(x|p) = \frac{x}{p} - \frac{1-x}{1-p}$$

    We obtain


    \begin{align*}
      \big(\frac{\partial}{\partial p} \log f(x|p) \big)^2 &= \big(\frac{x}{p} - \frac{1-x}{1-p}\big)^2\\
      &= \frac{x^2}{p^2} - \frac{2x(1-x)}{p(1-p)} + \frac{(1-x)^2}{(1-p)^2}\\
      &= \frac{x^2}{p^2} - \frac{2(x-x^2)}{p(1-p)} + \frac{(1-2x+x^2)}{(1-p)^2}
    \end{align*}

    Therefore,

    \begin{align*}
      I_x(p) &= E_p[(\frac{\partial}{\partial p} \log f(x|p))^2]\\
      &= \frac{p}{p^2} - \frac{2(p-p)}{p(1-p)} + \frac{1 - 2p + p}{(1-p)^2}\\
      &= \frac{1}{p} + \frac{1}{1-p}\\
      &= \frac{1}{p(1-p)}
    \end{align*}
  \end{field}
\end{note}




%%end_tag
%%start_tag Wk8

\begin{note}
  \begin{field}
    $$ I_x(\theta) = E_\theta\big[(\frac{\partial}{\partial \theta} \log f(x|\theta))^2\big] = $$

  \end{field}
  \begin{field}
    If $f(x|\theta)$ satisfies

    $$ \frac{\partial}{\partial \theta} E_\theta \big( \frac{\partial}{\partial \theta} \log f(x|\theta)\big) = \int \frac{\partial}{\partial \theta}\big[\frac{\partial}{\partial \theta}\log f(x|\theta)\big]f(x|\theta)dx $$

    $$ I_x(\theta) = E_\theta\big[(\frac{\partial}{\partial \theta} \log f(x|\theta))^2\big] = - E_\theta (\frac{\partial^2}{\partial \theta^2} \log f(x|\theta))$$
  \end{field}
\end{note}


\begin{note}
  \begin{field}
    Suppose that $X_1, \ldots , X_n$ are iid observations with common pdf or pmf $f(x|\theta)$. Then, the information about $\theta$ contained in the sample $\mathbf{X} = (X_1, \ldots , X_n)$ is
  \end{field}
  \begin{field}
    $$I_{\mathbf{X}}(\theta) = n I_{X_1}(\theta)$$
  \end{field}
\end{note}

\begin{note}
  \begin{field}
    Fisher Information - multiparameter case
  \end{field}
  \begin{field}
    Let $X$ be a random variable with pdf or pmf $f(x|\boldsymbol\theta)$, where $\boldsymbol\theta = (\theta_1,\theta_2) \in \Theta \subseteq \mathbb{R}^2$. Denote by $$I_{ij}(\boldsymbol\theta) = E_{\boldsymbol\theta}\big[(\frac{\partial}{\partial \theta_i}\log f(x|\boldsymbol\theta))(\frac{\partial}{\partial \theta_j} \log f(x|\boldsymbol\theta))\big] = -E_{\boldsymbol\theta}[\frac{\partial}{\partial \theta_i\theta_j}\log f(x|\boldsymbol\theta)]$$

    For $i,j = 1,2$.
    Then the (fisher) information matrix about $\boldsymbol\theta$ is

    $$ I_x(\boldsymbol\theta) = \begin{pmatrix}
      I_{11}(\boldsymbol\theta) & I_{12}(\boldsymbol\theta)\\
      I_{21}(\boldsymbol\theta) & I_{12}(\boldsymbol\theta)
    \end{pmatrix}$$
  \end{field}
\end{note}

\begin{note}
  \begin{field}
    Find Fisher information for Normal RVs
  \end{field}
  \begin{field}
    We have that $\boldsymbol\theta = (\mu,\sigma^2)$ and $f(x|\boldsymbol\theta) = \frac{1}{\sqrt{2\pi\sigma^2}}e^{-\frac{1}{2\sigma^2}(x - \mu)^2}$

    Then,
    $$ \frac{\partial}{\partial \mu} \log f(x|\boldsymbol\theta) = \frac{\partial}{\partial}[-\frac{1}{2}\log(2\pi\sigma^2) - \frac{1}{2\sigma^2}(x - \mu)^2] = \frac{(x-\mu)}{\sigma^2}$$

    $$ \frac{\partial}{\partial\sigma^2} = \frac{1}{2\sigma^2}[ \frac{(x-\mu)^2}{\sigma^2} - 1]$$

    Therefore $I_{11} = E_\theta[(\frac{\partial}{\partial \mu}\log f(x|\boldsymbol\theta))^2] = E_\theta[\frac{(x-\mu)^2}{\sigma^4}] = \frac{1}{\sigma^4}\sigma^2 = \frac{1}{\sigma^2}$

    \begin{align*}
      I_{22}(\boldsymbol\theta) &= E_{\theta}[\frac{\partial}{\partial\sigma^2}\log f(x|\boldsymbol\theta)^2]\\
      &= E_\theta \big\{ [ \frac{1}{2\sigma^2} ( \frac{(x-\mu)^2}{\sigma^2} - 1)]^2\big\}\\
      &= \frac{1}{4\sigma^4}E_\theta[ (\frac{(x-\mu)^2}{\sigma^2} - 1)^2]\\
      &= \frac{1}{4\sigma^4 \cdot 2} \\
      &= \frac{1}{2\sigma^4} \quad \text{Since } = V(\chi^2_1)
    \end{align*}

    Now for the off diagonal elements,

    \begin{align*}
      I_{12}(\boldsymbol\theta) = I_{22}(\boldsymbol\theta) &= E_\theta \big[ ( \frac{\partial{}{}}{\partial{\mu}{}}\log f(x|\theta)(\frac{\partial{}{}}{\partial{\sigma^2}{    }} \log f(x|\theta)))\big]\\
      &=E_\theta \big[ \frac{(x-\mu)}{\sigma^2
      } \frac{1}{2\sigma^2} [\frac{x-\mu}{\sigma^2}\cdot 1]\big]\\
      &= \frac{1}{2\sigma^4} E_\theta[ \frac{(x-\mu)^3}{\sigma^3} - (x-\mu)]
    \end{align*}

    But $E_\theta[(x-\mu)^3] = E_\theta[(x - \mu)] = 0$, because $X$ is symmetric around $\mu$, and we obtain $I_{12}(\boldsymbol\theta) = I_{21}(\boldsymbol\theta) = 0$

    We obtain that

    \begin{align*}
      I_{x_1}(\boldsymbol\theta) &= \begin{pmatrix}
        I_{11}(\boldsymbol\theta) & I_{12}(\boldsymbol\theta)\\
        I_{21}(\boldsymbol\theta) & I_{22}(\boldsymbol\theta)
    \end{pmatrix}\\
    &= \begin{pmatrix}
      \frac{1}{\sigma^2} & 0 \\ 0 & \frac{1}{2\sigma^4}
    \end{pmatrix}
    \end{align*}


    And hence $$ I_{\mathbf{x}}(\boldsymbol\theta) = n I_{X_1}(\boldsymbol\theta) = \begin{pmatrix}
      \frac{n}{\sigma^2} & 0 \\ 0 & \frac{n}{2\sigma^4}
    \end{pmatrix} $$
  \end{field}
\end{note}

\begin{note}
  \begin{field}
    $I_T(\theta) \leq $
  \end{field}
  \begin{field}
    $I_T(\theta) \leq I_{\mathbf{X}}(\theta)$
    (The information of the statistic is less than or equal to the information of the sample)
  \end{field}
\end{note}


\begin{note}
  \begin{field}
    Let $\mathbf{X} = X_1, \ldots , X_n$ denote the entire data, and let $T = T(\mathbf{X})$ be some statistic. Then, for all $\theta \in \Theta \subseteq \mathbb{R}$, $I_{\mathbf{X}} (\theta) \geq I_t(\theta)$
    Where the equality is attained...
  \end{field}
  \begin{field}
    Let $\mathbf{X} = X_1, \ldots , X_n$ denote the entire data, and let $T = T(\mathbf{X})$ be some statistic. Then, for all $\theta \in \Theta \subseteq \mathbb{R}$, $I_{\mathbf{X}} (\theta) \geq I_t(\theta)$
    Where the equality is attained if and only iff $T(\mathbf{X})$ is sufficient for $\theta$
  \end{field}
\end{note}

\begin{note}
  \begin{field}
    Let $\mathbf{X} = (X_1, \ldots , X_n)$, denote a sample of iid observations and suppose the statistic $T(\mathbf{X}) = (T_1(\mathbf{X}), T_2(\mathbf{X}))$ is such that $T_1$ and $T_2$ are independent. Then $$ I_{T}(\boldsymbol\theta) = $$
  \end{field}
  \begin{field}
    Let $\mathbf{X} = (X_1, \ldots , X_n)$, denote a sample of iid observations and suppose the statistic $T(\mathbf{X}) = (T_1(\mathbf{X}), T_2(\mathbf{X}))$ is such that $T_1$ and $T_2$ are independent. Then $$ I_{T}(\boldsymbol\theta) = I_{T_1}(\boldsymbol\theta) + I_{T_2}(\boldsymbol\theta)$$
  \end{field}
\end{note}

\begin{note}
  \begin{field}
    Point estimator
  \end{field}
  \begin{field}
    Any statistic $T(\mathbf{X})$ that is used to estimate the value of a parameter is called a point estimator of $\theta$. We write $\hat{\theta} = T(\mathbf{X})$
  \end{field}
\end{note}

\begin{note}
  \begin{field}
    Method of moments
  \end{field}
  \begin{field}
    \begin{align*}
      m_1 &= \frac{1}{n}\sum _{i = 1}^n X_i^1, \quad \mu_1 = E(X^1)\\
      m_2 &= \frac{1}{n}\sum _{i = 1}^n X_i^2, \quad \mu_2 = E(X^2)\\
      \vdots & \\
      m_k &= \frac{1}{n} \sum _{i = 1}^n X_i^k \quad \mu_k = E(X^k)
    \end{align*}
    Equating and solving for $\theta$ gives the MoM estimators
  \end{field}
\end{note}

\begin{note}
  \begin{field}
    Example Method of Moments
    Suppose that $X_1, \ldots , X_n$ are iid Binomial($k,p$), where both $k$ and $p$ are unknown.
  \end{field}
  \begin{field}
    We have that
    $$ P(X_i = x|k,p) = \binom{k}{x}p^x(1-p)^{k-x}, x = 0,1,\ldots,k$$
     and we obtain $E(X_1) = kp$, $E(X_1^2) = kp(1-p) + k^2p^2$

    Solving the sytem of equations we obtain

    \begin{align*}
      m_1 &= \frac{1}{n} \sum _{i = 1}^n X_i = kp\\
      m_2 &= \frac{1}{n} \sum _{i = 1}^n X_i^2 = kp(1-p) + k^2p^2
    \end{align*}

    Sovling the system of equations:

    \begin{align*}
      \tilde{p} &= \frac{\bar{x}}{\tilde{k}}\\
      \tilde{k} &= \frac{\bar{x}^2}{\bar{x} - \frac{1}{n} \sum _{i = 1}^n (x_i - \bar{x})^2}
    \end{align*}

    Possible problems: $k$ has to be an integer, and not negative. (Estimates of parameters that are outside of the parameter space. )
  \end{field}
  \begin{field}

  \end{field}
\end{note}


%%end_tag
%%start_tag Wk9

\begin{note}
  \begin{field}
    Maximum Likelihood Estimator
  \end{field}
  \begin{field}
    In this context, we define the \textbf{Maximum Likelihood Estimator (MLE)} of $\theta$ as the parameter value $\hat{\theta}_{ML} = \hat{\theta}(\mathbf{x})$ that satisfies
    $$L(\hat{\theta}_{ML}|\mathbf{x}) = \text{sup}_{\theta \in \Theta} L(\theta|\mathbf{x})$$

    Note this often proceedes as taking the derivative of the log likelihood function and setting to zero to solve for parameters - not always
  \end{field}
\end{note}

\begin{note}
  \begin{field}
    Example of MLE
    Suppose that $X_1, \ldots , X_n$ are iid Exponential$(\lambda)$. Find the MLE $\hat{\lambda}_{ML}$ of $\lambda$
  \end{field}
  \begin{field}
    Suppose that $X_1, \ldots , X_n$ are iid Exponential$(\lambda)$. Find the MLE $\hat{\lambda}_{ML}$ of $\lambda$

    We have that $f(x|\lambda) = \frac{1}{\lambda}e^{x/\lambda}$, $x>0$, and therefore

    $$L(\lambda|x) = \prod _{i = 1}^n \frac{1}{\lambda}e^{x_i/\lambda} = \lambda^{-n}e^{-\frac{1}{\lambda}\sum _{i = 1}^n x_i}$$


    Since $\log(\cdot)$ is a strictly monotone (one-to-one) and increasing, we consider instead the maximization of the log-likelihood

    $$l(\lambda|\mathbf{x}) = \log L(\lambda|\mathbf{x}) = -n\log \lambda - \frac{1}{\lambda} \sum _{i = 1}^n x_i$$

    $$\frac{\partial  }{\partial \lambda } l(\lambda | \mathbf{x}) = \frac{-n}{\lambda} + \frac{1}{\lambda^2}\sum _{i = 1}^n x_i$$

    Solving $\frac{\partial  }{\partial \lambda } l(\lambda|\mathbf{x})= 0$, we obtain

    $$ \frac{-n}{\lambda} + \frac{1}{\lambda^2}\sum _{i = 1}^n x_i = 0$$

    $$ -n\lambda + n \bar{x} = 0$$
    $$\lambda = \bar{x}$$

  \end{field}
\end{note}


\begin{note}
  \begin{field}
    Example of MLE when can't differemtiate

    Suppose that $X_1, \ldots , X_n$ are iid Uniform($0,\theta$), $\theta > 0$. Find the MLE of $\theta$
  \end{field}
  \begin{field}
    We have that $f(x|\theta) = \frac{1}{\theta}I(0<x<\theta)$

    And therefore
    \begin{align*}
      L(\theta|\mathbf{x}) &= \prod _{i = 1}^n \frac{1}{\theta} I(0 < x_i < \theta)\\
      &= \frac{1}{\theta^n} I(X_{(1)} > 0)I(X_{(n)}< \theta)
    \end{align*}

    In this case, the support of $X$ depends on $\theta$ and the maximization problem only makes sense whenever $L(\theta|\mathbf{x}) > 0$. We cannot simply approach the problem by taking partial derivatives, but assuming the likelihood is positive, we notice that $L(\theta|\mathbf{x})$ is decerasing as a function of $\theta$, for $\theta > X_{(n)}$

    Picture with $L(\theta)$ as zero untill $X_{(n)}$ on x axis, goes up to $1/X_{(n)}$ there and decreases with $\frac{1}{\theta^n}$

    It follows the MLE of $\theta$ is $\hat{\theta}_{ML} = X_{(n)}$

  \end{field}
\end{note}





%%end_tag

%%start_tag Wk10

\begin{note}
  \begin{field}
    If $\hat{\theta}_{ML}$ is the MLE of $\theta$, then for any function $\tau(\theta)$, the MLE of $\eta =\tau(\theta)$ is $\hat{\eta}_{ML} =$
  \end{field}
  \begin{field}
    If $\hat{\theta}_{ML}$ is the MLE of $\theta$, then for any function $\tau(\theta)$, the MLE of $\eta =\tau(\theta)$ is $\hat{\eta}_{ML} = \tau(\hat{\theta}_{ML})$
  \end{field}
\end{note}


\begin{note}
  \begin{field}
    Bias
  \end{field}
  \begin{field}
    Let $\hat{\theta} = T(\mathbf{X})$ be an estimatro of $\theta$. Then the Bias of $\hat{\theta}$ as an estimator of $\theta$ is defined as
    $$B_{\theta}(\hat{\theta}) = E_\theta(\hat{\theta} - \theta) = E_\theta(\hat{\theta}) - \theta$$

    That is the difference between the expected value of $\hat{\theta}$ and $\theta$.

    An estimator $\hat{\theta}$ of $\theta$ is said to be unbiased if $B_\theta(\hat{\theta}) = 0 \quad \forall \theta$
  \end{field}
\end{note}

\begin{note}
  \begin{field}
    Mean Squared Error
  \end{field}
  \begin{field}
    Let $\hat{\theta} = T(\mathbf{X})$ be an estimate of $\theta$. Then, the \textbf{Mean Squared Error} (MSE) of $\hat{\theta}$ as an estimator of $\theta$ is defined as:

    $$ MSE (\hat{\theta}) = E_{\theta}[(\hat{\theta} - \theta)^2] = V_\theta(\hat{\theta}) + [B_\theta(\hat{\theta})]^2$$

  \end{field}
\end{note}

\begin{note}
  \begin{field}
    Do unbiased estimators always exist?
  \end{field}
  \begin{field}
    No, Suppose that $X \sim $ Binomial($n,p$) and let $\theta = 1/p$ be the parameter of interest. Can we find an unbiased estimator for $\theta$?- No
  \end{field}
\end{note}

\begin{note}
  \begin{field}
    UMVUE
  \end{field}
  \begin{field}
    An estimator $W^*$ is called a best unbiased estimator of $\tau(\theta)$ if it satisfies $E_\theta(W^*) = \tau(\theta)$, for all $\theta$, and for any other estimator $W$ with $E_\theta(W) = \tau(\theta)$, we have $V_\theta(W^*)\leq V_\theta(W), \forall \theta$. Equivalently $W^*$ is also called a \textbf{Uniform Minimal Variance Unbiased Estimator} (UMVUE) of $\tau(\theta)$
  \end{field}
\end{note}

\begin{note}
  \begin{field}
    Finding a UMVUE
  \end{field}
  \begin{field}
    Start with a complete statistic, (find min suff statistic, prove completeness), Find bias (ie $E(T(\mathbf{X}))$). Then adjust $T(\mathbf{X})$ to be unbiased. (ie center or scale )
  \end{field}
\end{note}

\begin{note}
  \begin{field}
    Cramer-Rao Inequality
  \end{field}
  \begin{field}
    Let $X_1, \ldots , X_n$ be a sample with joint pdf or pmf $f(\mathbf{x}|\theta)$ and let $W(\mathbf{X}) = W(X_1, \ldots , X_n)$ be any estimator satisfying
    $$\frac{d}{d\theta} E_\theta(W(X)) = \int \frac{d}{d\theta} [W(\mathbf{X})f(\mathbf{x}|\theta)] d \mathbf{x}$$

    and $V_\theta(W(\mathbf{X}))< \infty$

    Then,
    $$V_\theta(W(\mathbf{X})) \geq \frac{(\frac{d}{d\theta}E_\theta(W(\mathbf{X})))^2}{E_\theta[( \frac{\partial}{\partial \theta}\log f(\mathbf{x}|\theta))^2]}$$

    Observe that if the sample $X_1, \ldots , X_n$ is iid with common pdf or pmf $f(x|\theta)$, we obtain

    $$ V_\theta (W(\mathbf{X})) \geq \frac{[\frac{d}{d\theta}E_\theta(W(\mathbf{X}))]^2}{nE_\theta[(\log f(\mathbf{x}|\theta))^2]}$$

    The denominator is the information in the sample about $\theta$

    We have that as the information number gets bigger we have a smaller bound for the variance. of the best unbiased estimator and therefore more information is available.
  \end{field}
\end{note}

\begin{note}
  \begin{field}
    Cramer-Rao and UMVUE example
    UMVUE of $\lambda$ for Poisson
  \end{field}
  \begin{field}
    Poisson example, we have $\tau(\lambda) = \lambda$, so $\frac{d}{d\lambda}\tau(\lambda) = 1$

    On the other hand,

    \begin{align*}
      nE_\lambda[(\frac{d}{d\lambda}\log f(x|\lambda))^2] &= -n E_\lambda(\frac{\partial ^2 }{\partial \lambda ^2}) \log f(x|\lambda))\\
      &= -n E_\lambda(\frac{\partial ^2 }{\partial \lambda ^2} \log (\frac{e^{-\lambda}\lambda^x}{x!}))\\
      &= -n E_\lambda[\frac{\partial ^2 }{\partial \lambda ^2}(-\lambda + x \log \lambda - \log(x!))]\\
      &= -n E_\lambda(\frac{-x}{\lambda^2})\\
      &= \frac{n}{\lambda}
    \end{align*}

    Therefore, for any unbiased estimator $W$ of $\lambda$, we must have $V_\lambda(W) \geq \lambda /n$. Since $V_\lambda(\bar{X}) = \frac{\lambda}{n}$, we have that $\bar{X}$ is an UMVUE of $\lambda$
  \end{field}
\end{note}

\begin{note}
  \begin{field}
    Does $S^t$ for Normal attain cramer rao?
  \end{field}
  \begin{field}
    No -
    Suppose that $X_1, \ldots , X_n$ are iid N($\mu,\sigma^2$)
    and consider the estimation of $\sigma^2$ when $\mu$ is unknown.

    We have that

    $$ \frac{\partial ^2 }{\partial (\sigma^2) ^2} \log [ \frac{1}{\sqrt{2\pi\sigma^2}} e^{-\frac{1}{2\sigma^2}(x-\mu)^2}] = \frac{1}{2\sigma^4} - \frac{(x-\mu)^2}{\sigma^6}$$

    and
    \begin{align*}
      -E[\frac{\partial ^2 }{\partial (\sigma^2)^2 } \log f(x|\mu,\sigma^2)] &= -E(\frac{1}{2\sigma^4} - \frac{(x-\mu)^2}{\sigma^6})\\
      &= -\frac{1}{2\sigma^4} + \frac{\sigma^2}{\sigma^6}\\
      &= \frac{1}{2\sigma^4}
    \end{align*}

    and therefore, any unbiased estimator $W$ of $\sigma^2$ must satisfy $V(W) \geq \frac{2\sigma^4}{n}$. Recall that for $S^2$ we have $$V(S^2) = \frac{2\sigma^4}{n-1} > \frac{2\sigma^4}{n}$$

    and therefore $S^2$ does not attain the cramer-rao lower bound.

  \end{field}
\end{note}

\begin{note}
  \begin{field}
    Rao-Blackwell
  \end{field}
  \begin{field}
    Let $W$ be any unbiased estimator $\tau(\theta)$ and let $T$ be a sufficient statistic for $\theta$. Define $\phi(T) = E(W|T)$. Then $E_\theta(\phi(T)) = \tau(\theta)$ and $V_\theta(\phi(T))\leq V_\theta(W)$, for all $\theta$
    That is, $\phi(T)$ is a uniformly better unbiased estimator of $\tau(\theta)$
  \end{field}
\end{note}

\begin{note}
  \begin{field}
    Use of Rao-Blackwell
  \end{field}
  \begin{field}
    Estimators can be improved (their MSE) using sufficiency (already sufficient statistics, or functions of sufficient statistics cannot be improved)
  \end{field}
\end{note}

\begin{note}
  \begin{field}
    Are unbiased estimators based on complete sufficient statistics unique.
  \end{field}
  \begin{field}
    Unbiased estimators based on complete sufficient statistics are unique.
  \end{field}
\end{note}






%%end_tag


\end{document}
